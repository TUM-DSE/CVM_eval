\documentclass[letterpaper,twocolumn,10pt]{article}
\usepackage{usenix-2020-09}

\newcommand{\titlename}{
Evaluating Confidential Computing with Unikernels \\ (Guided Research Project)
}

\newcommand{\authorname}{}

\usepackage[english]{babel}
\usepackage[backend=biber]{biblatex}
\usepackage{booktabs}
\usepackage{graphicx}
\usepackage{caption}
\usepackage{subcaption}
\usepackage{tikz}
\usepackage{amsmath}
\usepackage{listings}
\usepackage{multicol}
%\usepackage{flushend}
%\usepackage[hyphens]{url}
\PassOptionsToPackage{hyphens}{url}
\usepackage{hyperref}
\hypersetup{
    colorlinks=true,
    allcolors=blue,
    breaklinks=true,
    bookmarksnumbered=true,
    bookmarkstype=toc,
    bookmarksopen=true,
    hidelinks,
    pdftitle={\titlename},
    pdfauthor={\authorname},
    pdfstartview={FitH -32768}
}

% suppress hyphenation
\hyphenpenalty=1000\relax
\exhyphenpenalty=1000\relax
\sloppy

% font
\usepackage{amsmath,amssymb,amsfonts}
\usepackage{libertine}
\usepackage{libertinust1math}
\usepackage{inconsolata}

% lsting
\usepackage{xcolor}
\definecolor{codegreen}{rgb}{0,0.6,0}
\definecolor{codegray}{rgb}{0.5,0.5,0.5}
\definecolor{codepurple}{rgb}{0.58,0,0.82}
\definecolor{backcolour}{rgb}{0.95,0.95,0.92}
\lstdefinestyle{mystyle}{
    backgroundcolor=\color{backcolour},
    commentstyle=\color{codegreen},
    keywordstyle=\color{magenta},
    numberstyle=\tiny\color{codegray},
    stringstyle=\color{codepurple},
    basicstyle=\ttfamily\footnotesize,
    breakatwhitespace=false,
    breaklines=true,
    captionpos=b,
    keepspaces=true,
    numbers=left,
    numbersep=5pt,
    showspaces=false,
    showstringspaces=false,
    showtabs=false,
    tabsize=2
}
\lstset{style=mystyle}

%\titleformat{\paragraph}[runin]{\vspace{-2pt}\bf}{}{}{\periodafter}
\newcommand{\myparagraph}{\paragraph}
\newcommand{\MP}[1]{\textcolor{red}{#1}}
\newcommand{\grumbler}[2]{\MP{{\bf #1}: #2}}
\newcommand{\note}[1]{\grumbler{NOTE}{#1}}

% bibtex
\setcounter{biburllcpenalty}{7000}
\setcounter{biburlucpenalty}{8000}
\bibliography{reference}

% watermark
\usepackage{draftwatermark}
\SetWatermarkColor[gray]{0.9}

\begin{document}

%don't want date printed
\date{}

% make title bold and 14 pt font (Latex default is non-bold, 16 pt)
\title{\Large \bf \titlename}

%for single author (just remove % characters)
\author{
{\rm Roberto Castellotti}\\
TU Munich
\and
{\rm Masanori Misono}\\
TU Munich
%Second Institution
% copy the following lines to add more authors
\and
%Name Institution
} % end author

\maketitle

\begin{abstract}
We report a preliminary performance evaluation of Intel TDX (Trusted Domain Extension)~\cite{tdx} on a Linux system.
\end{abstract}

\section{Environment}

We use Intel DevCloud with Sapphire Rapids CPUs as the host machine and tdx-tools-2023ww01.rdc as the software stack.
\autoref{tab:experiment-environment} shows the detailed environment.
We use QEMU/KVM as a hypervisor.
We assign the guest the same amount of CPUs and about 90\% of the host memory.
We do not pin vCPU threads.
We disable vNUMA in the guest and boot a TDX VM with Direct Boot.
The guest uses kvmclock when TDX is disabled.
TDX VM does not use kvmclock as the guest cannot trust the host, but the TDX module provides reliable TSC~\cite{tdx_secure_spec}.

\begin{table*}[t]
\centering
\caption{Experiment environment}
\label{tab:experiment-environment}
\begin{tabular}{l|l}
\toprule
    Host CPU      & Intel(R) Xeon(R) Platinum 8480CTDX at 2.0GHz, 56 cores $\times$ 2 (Hyperthreading disabled) \\
    Host Memory   & Samsung DDR5 4800 MT/s 64 GB $\times$ 16 (1024GB) \\
    Host Config   & Automatic numa balancing disabled; Side channel mitigation default (enabled) \\
    Host Kernel   & 5.19.0-tdx.v2.4.mvp17.el8 (Ubuntu 22.04) \\
    Host TSC Frequency & 2000MHz \\
    QEMU          & 7.0-v1.3 (patched) \\
\midrule
    OVMF          & Stable 202211 (patched) \\
    Guest vCPU    & 112 \\
    Guest Memory  & 896GB  \\
    Guest Kernel  & 5.19.0-mvp15v2+4-generic (Ubuntu 22.04) \\
    Guest Config  & No vNUMA; Side channel mitigation default (enabled) \\
    TDX TSC Frequency & 1000MHz \\
\bottomrule
\end{tabular}
\end{table*}

\section{Micro Benchmarks}
\subsection{CPUID latency}
lorem ipsum


\begin{table}
\centering
\caption{cpuid leaf information}
\label{tab:cpuid}
\begin{tabular}{lll}
\toprule
leaf &  description & \#VE in TDX \\
\midrule
0x0  & vendor info & \\
0x2  & cache info  & \checkmark \\
0x15 & TSC info (trusted)   & \\
0x16 & TSC info    & \checkmark \\
\bottomrule
\end{tabular}
\end{table}

\begin{figure*}[t]
\centering
\includegraphics[width=1.0\textwidth]{./experiment/cpuid/cpuid_time.pdf}
% \includegraphics[width=1.0\textwidth]{./experiment/cpuid/cpuid_time_box.pdf}
\caption{CPUID latency distribution}
\label{fig:cpuid-latency}
\end{figure*}

\begin{table}
\centering
\caption{cpuid latency leaf: 0x0 (time: us)}
\label{tab:cpuid_0x0}
\begin{tabular}{lrrrr}
\toprule
{} &  50\% &  95\% &  99\% &   max \\
\midrule
bare            & 0.05 & 0.05 & 0.05 &  0.09 \\
bare:tme        & 0.04 & 0.05 & 0.05 &  0.09 \\
bare:tme:bypass & 0.04 & 0.05 & 0.05 &  6.06 \\
vm:bare         & 0.61 & 0.62 & 0.62 & 27.94 \\
vm:notdx        & 0.62 & 0.62 & 0.64 &  7.83 \\
vm:notdx:bypass & 0.62 & 0.62 & 0.63 &  4.11 \\
vm:tdx          & 1.45 & 1.46 & 1.47 & 31.95 \\
vm:tdx:bypass   & 1.49 & 1.62 & 1.63 & 29.64 \\
\bottomrule
\end{tabular}
\end{table}

\begin{table}
\centering
\caption{cpuid latency leaf: 0x2 (time: us)}
\label{tab:cpuid_0x2}
\begin{tabular}{lrrrr}
\toprule
{} &  50\% &  95\% &  99\% &   max \\
\midrule
bare            & 0.43 & 0.43 & 0.44 &  5.04 \\
bare:tme        & 0.42 & 0.43 & 0.44 &  5.89 \\
bare:tme:bypass & 0.42 & 0.43 & 0.43 &  5.79 \\
vm:bare         & 0.62 & 0.62 & 0.63 &  7.28 \\
vm:notdx        & 0.62 & 0.63 & 0.63 &  7.25 \\
vm:notdx:bypass & 0.62 & 0.62 & 0.64 &  8.28 \\
vm:tdx          & 3.59 & 3.61 & 3.62 & 74.86 \\
vm:tdx:bypass   & 3.65 & 4.75 & 4.79 & 30.05 \\
\bottomrule
\end{tabular}
\end{table}

\begin{table}
\centering
\caption{cpuid latency leaf: 0x15 (time: us)}
\label{tab:cpuid_0x15}
\begin{tabular}{lrrrr}
\toprule
{} &  50\% &  95\% &  99\% &   max \\
\midrule
bare            & 0.99 & 1.02 & 1.02 & 33.42 \\
bare:tme        & 0.99 & 1.02 & 1.02 &  6.51 \\
bare:tme:bypass & 0.98 & 1.00 & 1.01 &  9.90 \\
vm:bare         & 0.64 & 0.64 & 0.65 & 19.86 \\
vm:notdx        & 0.63 & 0.64 & 0.64 &  8.40 \\
vm:notdx:bypass & 0.64 & 0.65 & 0.65 &  7.55 \\
vm:tdx          & 1.47 & 1.60 & 1.61 & 48.42 \\
vm:tdx:bypass   & 1.49 & 1.51 & 1.52 & 43.69 \\
\bottomrule
\end{tabular}
\end{table}

\begin{table}
\centering
\caption{cpuid latency leaf: 0x16 (time: us)}
\label{tab:cpuid_0x16}
\begin{tabular}{lrrrr}
\toprule
{} &  50\% &  95\% &  99\% &   max \\
\midrule
bare            & 0.95 & 0.97 & 0.97 &  8.75 \\
bare:tme        & 0.97 & 0.99 & 0.99 &  6.46 \\
bare:tme:bypass & 0.96 & 0.98 & 0.98 &  6.50 \\
vm:bare         & 0.63 & 0.64 & 0.65 &  6.20 \\
vm:notdx        & 0.64 & 0.65 & 0.66 & 13.29 \\
vm:notdx:bypass & 0.64 & 0.64 & 0.65 &  4.51 \\
vm:tdx          & 3.58 & 3.61 & 3.62 & 62.92 \\
vm:tdx:bypass   & 3.65 & 3.67 & 3.68 & 47.16 \\
\bottomrule
\end{tabular}
\end{table}

\subsection{Memory overhead}
We measure the memory overhead of TDX using the following benchmarks using phoronix-test-suite~\cite{phoronix}.
Here, we report normalized overhead compared to the baremetal (``bare'').

\begin{description}
\item[RAMSpeed~\cite{ramspeed}] This measures the memory latency with several operations. \autoref{fig:ramspeed} shows the results.
\item[Tinymembench~\cite{tinymembench}] This benchmark measures the memory latency of the system. \autoref{fig:membench} shows the results.
\item[MBW~\cite{mbw}] This measures the memory bandwidth of the system. \autoref{fig:membench} shows the results.
% \item[Stream~\cite{stream}] This benchmark measures the memory bandwidth of the system. \autoref{fig:stream} shows the results.
\end{description}

\begin{figure*}[t]
\centering
\includegraphics[width=1.0\textwidth]{./experiment/phoronix/all_ramspeed.pdf}
\caption{RAMSpeed benchmarks (baseline: ``bare'')}
\label{fig:ramspeed}
\end{figure*}

\begin{figure*}[t]
\centering
\includegraphics[width=1.0\textwidth]{./experiment/phoronix/all_memory.pdf}
\caption{Tinymembench and MBW benchmarks (baseline: ``bare'')}
\label{fig:membench}
\end{figure*}

% TODO: we should check if vector instruction is available in VMs
% \begin{figure*}[t]
% \centering
% \includegraphics[width=1.0\textwidth]{./experiment/phoronix/all_stream.pdf}
% \caption{Stream benchmark (baseline: ``bare'')}
% \label{fig:stream}
% \end{figure*}

We observe the followings from the results.
\begin{itemize}
    \item For the RAMSpeed benchmarks, we observe 3.3\% overhead for ``bare:tme'' and 6.38\% for ``vm:tdx'' in geometric mean.
    \item For the Tinymembench benchmarks, we observe 5.95\% overhead for ``bare:tme'' and 4.42\% for ``vm:tdx'' in geometric mean.
    \item For the MBW benchmarks, we observe 9.37\% overhead for ``bare:tme'' and 10.52\% for ``vm:tdx'' in geometric mean.
    \item The overhead of the memory bandwidth (MBW) is larger than the overhead of the memory latency (RAMSpeed, Tinymembench).
\end{itemize}

\section{Application Benchmarks}
\label{sec:app:benchmark}

We measure several application benchmarks using Phoronix Benchmark Suite~\cite{phoronix}.
We especially run compilation and NPB (NAS Parallel Benchmarks) benchmarks as CPU-intensive applications and lz4 and SQLite benchmarks as memory-intensive applications.
Here, we report normalized overhead compared to the normal virtual machine (``bare:vm'').
\note{TDX VM (``vm:tdx'') may have additional overhead due to the vCPU over-commitment.}

\begin{description}
\item[Compilation benchmarks~\cite{compilation}] This measures compilation times of several applications. \autoref{fig:compilation} shows the results.
\item[NAS parallel benchmarks (NPB)~\cite{npb}] This measures the times of several MPI parallel applications. \autoref{fig:npb} shows the results.
\item[LZ4~\cite{lz4}] This measures the compression and decompression time with LZ4 algorithm. \autoref{fig:lz4} shows the results.
\item[SQLite~\cite{sqlite_bench}] This measures the time to perform a pre-defined number of insertions to a SQLite database. \autoref{fig:membench} shows the results.
\end{description}

\begin{figure*}[t]
\centering
\includegraphics[width=1.0\textwidth]{./experiment/phoronix/vm_compilation.pdf}
\caption{Compilation time (baseline: ``vm:bare'')}
\label{fig:compilation}
\end{figure*}

\begin{figure*}[t]
\centering
\includegraphics[width=1.0\textwidth]{./experiment/phoronix/vm_nas.pdf}
\caption{NAS Benchmarks (baseline: ``vm:bare'')}
\label{fig:npb}
\end{figure*}

\begin{figure*}[t]
\centering
\includegraphics[width=1.0\textwidth]{./experiment/phoronix/vm_lz4.pdf}
\caption{LZ4  (baseline: ``vm:bare'')}
\label{fig:lz4}∏
\end{figure*}

\begin{figure*}[t]
\centering
\includegraphics[width=1.0\textwidth]{./experiment/phoronix/vm_sqlite.pdf}
\caption{SQLite  (baseline: ``vm:bare'')}
\label{fig:sqlite}
\end{figure*}

We observe the followings from the results.
\begin{itemize}
    \item As of compilation and NPB benchmarks, we observe around 10 to up to 60\% overhead in the TDX VM (``vm:tdx''). However, vPCU over-commitment might affect these results, so we expect the actual performance will be better.
    \item As of LZ4 benchmaks, both ``vm:notdx'' and ``vm:tdx'' have similar performance. This is because LZ4 is a memory-intensive application, and the main overhead comes from memory encryption/decryption. NPB and these results also highlight the importance of TME bypass if we want to eliminate the memory encryption overhead in non-TDX VMs.
    \item As of SQLite benchmarks, we observe larger performance overhead in ``vm:tdx'' when copy size is larger than 32. This might be due to the vCPU over-commitment, but further investigation is needed.
\end{itemize}

\section{Remarks}
This report presents the basic performance experiments on Intel TDX with QEMU/KVM.
Further experiments will include but not be limited to the following.

\begin{itemize}
    \item More detailed breakdown of the overhead in TDX VM, including \#VE and TDX call handling.
    \item Performance evaluation on other hypervisors such as Cloud Hypervisor.
    \item Effect of vNUMA. vNUMA is known to have a significant impact on the performance of VM~\cite{sev_eval}.
    \item Memory management time in VMs.
    \item Performance effect of having multiple TDX and Non-TDX VMs.
    \item Boot time. TDX requires a special memory configuration, which affects the boot time.
    \item I/O performance with and without TDX-IO.
    \item Attestation time.
    \item Migration time.
\end{itemize}


%-------------------------------------------------------------------------------
% \bibliographystyle{plain}
\printbibliography

\appendix
\section*{Appendix}
\section{Errata and remarks on the documentation~\cite{linux-stacks-for-intel-tdx-2023ww01}}
\myparagraph{Section 3.2 ``BIOS Configurations''}
\begin{itemize}
    \item In our machine, we need to enable ``Socket Configuration => Processor Configuration => Extended APIC''. Otherwise, we could not enable TDX and SGX.
    \item In our machine, ``Disable excluding Mem below 1MB'' configuration is in ``Processor Configuration => TME, TME-MT, TDX => Disable excluding Mem below 1MB'', not "Processor Configuration => Disable excluding Mem below 1MB''.
    \item Also, the documentation says, ``Disable excluding Mem below 1MB in CMR'' should be ``Enable/Disable''. It is unclear whether we should enable it or not.
\end{itemize}

\myparagraph{Section 3.3.3 ``Ubuntu 22.04''}
\begin{itemize}
    \item `python-dev` is also a prerequisite. Otherwise, building mvp-tdx-kernel-v5.19 fails because of a lack of `python-config`.
\end{itemize}

\myparagraph{Section 3.5.1.2 ``Ubuntu 22.04''}
\begin{itemize}
    \item `libguestfs-tools` is also a prerequisite.
    \item We need to run `sudo chmod +r /boot/vmlinuz-*` before  running ./tdx-guest-stack.sh because in the shell script libguestfs tries to access /boot/vmlinuz, but that is not readable by default in Ubuntu (\url{https://bugs.launchpad.net/fuel/+bug/1467579})
\end{itemize}

\myparagraph{Section 5.4 ``Measurement \& EventLog Tools''}
\begin{itemize}
    \item The scripts are in the directory `tdx-tools-2023ww01.rdc/attestation/pytdxmeasure'.
\end{itemize}

\section{Troubleshooting}
Section 7.1 in the documentation~\cite{linux-stacks-for-intel-tdx-2023ww01} describes how to troubleshoot TDX.
Here, we describe some problems we encountered and how to solve them.

\myparagraph{Problem: TDX is not enabled (MSR reports TDX disabled) even if it is enabled in BIOS}
As the documentation says, ``at least 1 DIMM per CPU socket and ensure DIMM are plugged in corresponding slots.''
Inserting DIMMs properly solved the problem.

\myparagraph{Check MSR values}
We can check related MSR values with the following script.
TODO: insert it
\lstinputlisting[language=c]{./scripts/check_msr.sh}

\end{document}


% // this very basic program checks whether AMD SEV features are enabled
% // we are extracting data from register 0x8000001f[EAX], as specified in 
% // paragraphs 7.10.1,15.34.1,15.35.1,15.36.1 of AMD64 Architecture Programmer’s Manual
% // amd.com/system/files/TechDocs/40332.pdf
